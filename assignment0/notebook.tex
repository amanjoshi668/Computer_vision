
% Default to the notebook output style

    


% Inherit from the specified cell style.




    
\documentclass[11pt]{article}

    
    
    \usepackage[T1]{fontenc}
    % Nicer default font (+ math font) than Computer Modern for most use cases
    \usepackage{mathpazo}

    % Basic figure setup, for now with no caption control since it's done
    % automatically by Pandoc (which extracts ![](path) syntax from Markdown).
    \usepackage{graphicx}
    % We will generate all images so they have a width \maxwidth. This means
    % that they will get their normal width if they fit onto the page, but
    % are scaled down if they would overflow the margins.
    \makeatletter
    \def\maxwidth{\ifdim\Gin@nat@width>\linewidth\linewidth
    \else\Gin@nat@width\fi}
    \makeatother
    \let\Oldincludegraphics\includegraphics
    % Set max figure width to be 80% of text width, for now hardcoded.
    \renewcommand{\includegraphics}[1]{\Oldincludegraphics[width=.8\maxwidth]{#1}}
    % Ensure that by default, figures have no caption (until we provide a
    % proper Figure object with a Caption API and a way to capture that
    % in the conversion process - todo).
    \usepackage{caption}
    \DeclareCaptionLabelFormat{nolabel}{}
    \captionsetup{labelformat=nolabel}

    \usepackage{adjustbox} % Used to constrain images to a maximum size 
    \usepackage{xcolor} % Allow colors to be defined
    \usepackage{enumerate} % Needed for markdown enumerations to work
    \usepackage{geometry} % Used to adjust the document margins
    \usepackage{amsmath} % Equations
    \usepackage{amssymb} % Equations
    \usepackage{textcomp} % defines textquotesingle
    % Hack from http://tex.stackexchange.com/a/47451/13684:
    \AtBeginDocument{%
        \def\PYZsq{\textquotesingle}% Upright quotes in Pygmentized code
    }
    \usepackage{upquote} % Upright quotes for verbatim code
    \usepackage{eurosym} % defines \euro
    \usepackage[mathletters]{ucs} % Extended unicode (utf-8) support
    \usepackage[utf8x]{inputenc} % Allow utf-8 characters in the tex document
    \usepackage{fancyvrb} % verbatim replacement that allows latex
    \usepackage{grffile} % extends the file name processing of package graphics 
                         % to support a larger range 
    % The hyperref package gives us a pdf with properly built
    % internal navigation ('pdf bookmarks' for the table of contents,
    % internal cross-reference links, web links for URLs, etc.)
    \usepackage{hyperref}
    \usepackage{longtable} % longtable support required by pandoc >1.10
    \usepackage{booktabs}  % table support for pandoc > 1.12.2
    \usepackage[inline]{enumitem} % IRkernel/repr support (it uses the enumerate* environment)
    \usepackage[normalem]{ulem} % ulem is needed to support strikethroughs (\sout)
                                % normalem makes italics be italics, not underlines
    

    
    
    % Colors for the hyperref package
    \definecolor{urlcolor}{rgb}{0,.145,.698}
    \definecolor{linkcolor}{rgb}{.71,0.21,0.01}
    \definecolor{citecolor}{rgb}{.12,.54,.11}

    % ANSI colors
    \definecolor{ansi-black}{HTML}{3E424D}
    \definecolor{ansi-black-intense}{HTML}{282C36}
    \definecolor{ansi-red}{HTML}{E75C58}
    \definecolor{ansi-red-intense}{HTML}{B22B31}
    \definecolor{ansi-green}{HTML}{00A250}
    \definecolor{ansi-green-intense}{HTML}{007427}
    \definecolor{ansi-yellow}{HTML}{DDB62B}
    \definecolor{ansi-yellow-intense}{HTML}{B27D12}
    \definecolor{ansi-blue}{HTML}{208FFB}
    \definecolor{ansi-blue-intense}{HTML}{0065CA}
    \definecolor{ansi-magenta}{HTML}{D160C4}
    \definecolor{ansi-magenta-intense}{HTML}{A03196}
    \definecolor{ansi-cyan}{HTML}{60C6C8}
    \definecolor{ansi-cyan-intense}{HTML}{258F8F}
    \definecolor{ansi-white}{HTML}{C5C1B4}
    \definecolor{ansi-white-intense}{HTML}{A1A6B2}

    % commands and environments needed by pandoc snippets
    % extracted from the output of `pandoc -s`
    \providecommand{\tightlist}{%
      \setlength{\itemsep}{0pt}\setlength{\parskip}{0pt}}
    \DefineVerbatimEnvironment{Highlighting}{Verbatim}{commandchars=\\\{\}}
    % Add ',fontsize=\small' for more characters per line
    \newenvironment{Shaded}{}{}
    \newcommand{\KeywordTok}[1]{\textcolor[rgb]{0.00,0.44,0.13}{\textbf{{#1}}}}
    \newcommand{\DataTypeTok}[1]{\textcolor[rgb]{0.56,0.13,0.00}{{#1}}}
    \newcommand{\DecValTok}[1]{\textcolor[rgb]{0.25,0.63,0.44}{{#1}}}
    \newcommand{\BaseNTok}[1]{\textcolor[rgb]{0.25,0.63,0.44}{{#1}}}
    \newcommand{\FloatTok}[1]{\textcolor[rgb]{0.25,0.63,0.44}{{#1}}}
    \newcommand{\CharTok}[1]{\textcolor[rgb]{0.25,0.44,0.63}{{#1}}}
    \newcommand{\StringTok}[1]{\textcolor[rgb]{0.25,0.44,0.63}{{#1}}}
    \newcommand{\CommentTok}[1]{\textcolor[rgb]{0.38,0.63,0.69}{\textit{{#1}}}}
    \newcommand{\OtherTok}[1]{\textcolor[rgb]{0.00,0.44,0.13}{{#1}}}
    \newcommand{\AlertTok}[1]{\textcolor[rgb]{1.00,0.00,0.00}{\textbf{{#1}}}}
    \newcommand{\FunctionTok}[1]{\textcolor[rgb]{0.02,0.16,0.49}{{#1}}}
    \newcommand{\RegionMarkerTok}[1]{{#1}}
    \newcommand{\ErrorTok}[1]{\textcolor[rgb]{1.00,0.00,0.00}{\textbf{{#1}}}}
    \newcommand{\NormalTok}[1]{{#1}}
    
    % Additional commands for more recent versions of Pandoc
    \newcommand{\ConstantTok}[1]{\textcolor[rgb]{0.53,0.00,0.00}{{#1}}}
    \newcommand{\SpecialCharTok}[1]{\textcolor[rgb]{0.25,0.44,0.63}{{#1}}}
    \newcommand{\VerbatimStringTok}[1]{\textcolor[rgb]{0.25,0.44,0.63}{{#1}}}
    \newcommand{\SpecialStringTok}[1]{\textcolor[rgb]{0.73,0.40,0.53}{{#1}}}
    \newcommand{\ImportTok}[1]{{#1}}
    \newcommand{\DocumentationTok}[1]{\textcolor[rgb]{0.73,0.13,0.13}{\textit{{#1}}}}
    \newcommand{\AnnotationTok}[1]{\textcolor[rgb]{0.38,0.63,0.69}{\textbf{\textit{{#1}}}}}
    \newcommand{\CommentVarTok}[1]{\textcolor[rgb]{0.38,0.63,0.69}{\textbf{\textit{{#1}}}}}
    \newcommand{\VariableTok}[1]{\textcolor[rgb]{0.10,0.09,0.49}{{#1}}}
    \newcommand{\ControlFlowTok}[1]{\textcolor[rgb]{0.00,0.44,0.13}{\textbf{{#1}}}}
    \newcommand{\OperatorTok}[1]{\textcolor[rgb]{0.40,0.40,0.40}{{#1}}}
    \newcommand{\BuiltInTok}[1]{{#1}}
    \newcommand{\ExtensionTok}[1]{{#1}}
    \newcommand{\PreprocessorTok}[1]{\textcolor[rgb]{0.74,0.48,0.00}{{#1}}}
    \newcommand{\AttributeTok}[1]{\textcolor[rgb]{0.49,0.56,0.16}{{#1}}}
    \newcommand{\InformationTok}[1]{\textcolor[rgb]{0.38,0.63,0.69}{\textbf{\textit{{#1}}}}}
    \newcommand{\WarningTok}[1]{\textcolor[rgb]{0.38,0.63,0.69}{\textbf{\textit{{#1}}}}}
    
    
    % Define a nice break command that doesn't care if a line doesn't already
    % exist.
    \def\br{\hspace*{\fill} \\* }
    % Math Jax compatability definitions
    \def\gt{>}
    \def\lt{<}
    % Document parameters
    \title{Question3}
    
    
    

    % Pygments definitions
    
\makeatletter
\def\PY@reset{\let\PY@it=\relax \let\PY@bf=\relax%
    \let\PY@ul=\relax \let\PY@tc=\relax%
    \let\PY@bc=\relax \let\PY@ff=\relax}
\def\PY@tok#1{\csname PY@tok@#1\endcsname}
\def\PY@toks#1+{\ifx\relax#1\empty\else%
    \PY@tok{#1}\expandafter\PY@toks\fi}
\def\PY@do#1{\PY@bc{\PY@tc{\PY@ul{%
    \PY@it{\PY@bf{\PY@ff{#1}}}}}}}
\def\PY#1#2{\PY@reset\PY@toks#1+\relax+\PY@do{#2}}

\expandafter\def\csname PY@tok@w\endcsname{\def\PY@tc##1{\textcolor[rgb]{0.73,0.73,0.73}{##1}}}
\expandafter\def\csname PY@tok@c\endcsname{\let\PY@it=\textit\def\PY@tc##1{\textcolor[rgb]{0.25,0.50,0.50}{##1}}}
\expandafter\def\csname PY@tok@cp\endcsname{\def\PY@tc##1{\textcolor[rgb]{0.74,0.48,0.00}{##1}}}
\expandafter\def\csname PY@tok@k\endcsname{\let\PY@bf=\textbf\def\PY@tc##1{\textcolor[rgb]{0.00,0.50,0.00}{##1}}}
\expandafter\def\csname PY@tok@kp\endcsname{\def\PY@tc##1{\textcolor[rgb]{0.00,0.50,0.00}{##1}}}
\expandafter\def\csname PY@tok@kt\endcsname{\def\PY@tc##1{\textcolor[rgb]{0.69,0.00,0.25}{##1}}}
\expandafter\def\csname PY@tok@o\endcsname{\def\PY@tc##1{\textcolor[rgb]{0.40,0.40,0.40}{##1}}}
\expandafter\def\csname PY@tok@ow\endcsname{\let\PY@bf=\textbf\def\PY@tc##1{\textcolor[rgb]{0.67,0.13,1.00}{##1}}}
\expandafter\def\csname PY@tok@nb\endcsname{\def\PY@tc##1{\textcolor[rgb]{0.00,0.50,0.00}{##1}}}
\expandafter\def\csname PY@tok@nf\endcsname{\def\PY@tc##1{\textcolor[rgb]{0.00,0.00,1.00}{##1}}}
\expandafter\def\csname PY@tok@nc\endcsname{\let\PY@bf=\textbf\def\PY@tc##1{\textcolor[rgb]{0.00,0.00,1.00}{##1}}}
\expandafter\def\csname PY@tok@nn\endcsname{\let\PY@bf=\textbf\def\PY@tc##1{\textcolor[rgb]{0.00,0.00,1.00}{##1}}}
\expandafter\def\csname PY@tok@ne\endcsname{\let\PY@bf=\textbf\def\PY@tc##1{\textcolor[rgb]{0.82,0.25,0.23}{##1}}}
\expandafter\def\csname PY@tok@nv\endcsname{\def\PY@tc##1{\textcolor[rgb]{0.10,0.09,0.49}{##1}}}
\expandafter\def\csname PY@tok@no\endcsname{\def\PY@tc##1{\textcolor[rgb]{0.53,0.00,0.00}{##1}}}
\expandafter\def\csname PY@tok@nl\endcsname{\def\PY@tc##1{\textcolor[rgb]{0.63,0.63,0.00}{##1}}}
\expandafter\def\csname PY@tok@ni\endcsname{\let\PY@bf=\textbf\def\PY@tc##1{\textcolor[rgb]{0.60,0.60,0.60}{##1}}}
\expandafter\def\csname PY@tok@na\endcsname{\def\PY@tc##1{\textcolor[rgb]{0.49,0.56,0.16}{##1}}}
\expandafter\def\csname PY@tok@nt\endcsname{\let\PY@bf=\textbf\def\PY@tc##1{\textcolor[rgb]{0.00,0.50,0.00}{##1}}}
\expandafter\def\csname PY@tok@nd\endcsname{\def\PY@tc##1{\textcolor[rgb]{0.67,0.13,1.00}{##1}}}
\expandafter\def\csname PY@tok@s\endcsname{\def\PY@tc##1{\textcolor[rgb]{0.73,0.13,0.13}{##1}}}
\expandafter\def\csname PY@tok@sd\endcsname{\let\PY@it=\textit\def\PY@tc##1{\textcolor[rgb]{0.73,0.13,0.13}{##1}}}
\expandafter\def\csname PY@tok@si\endcsname{\let\PY@bf=\textbf\def\PY@tc##1{\textcolor[rgb]{0.73,0.40,0.53}{##1}}}
\expandafter\def\csname PY@tok@se\endcsname{\let\PY@bf=\textbf\def\PY@tc##1{\textcolor[rgb]{0.73,0.40,0.13}{##1}}}
\expandafter\def\csname PY@tok@sr\endcsname{\def\PY@tc##1{\textcolor[rgb]{0.73,0.40,0.53}{##1}}}
\expandafter\def\csname PY@tok@ss\endcsname{\def\PY@tc##1{\textcolor[rgb]{0.10,0.09,0.49}{##1}}}
\expandafter\def\csname PY@tok@sx\endcsname{\def\PY@tc##1{\textcolor[rgb]{0.00,0.50,0.00}{##1}}}
\expandafter\def\csname PY@tok@m\endcsname{\def\PY@tc##1{\textcolor[rgb]{0.40,0.40,0.40}{##1}}}
\expandafter\def\csname PY@tok@gh\endcsname{\let\PY@bf=\textbf\def\PY@tc##1{\textcolor[rgb]{0.00,0.00,0.50}{##1}}}
\expandafter\def\csname PY@tok@gu\endcsname{\let\PY@bf=\textbf\def\PY@tc##1{\textcolor[rgb]{0.50,0.00,0.50}{##1}}}
\expandafter\def\csname PY@tok@gd\endcsname{\def\PY@tc##1{\textcolor[rgb]{0.63,0.00,0.00}{##1}}}
\expandafter\def\csname PY@tok@gi\endcsname{\def\PY@tc##1{\textcolor[rgb]{0.00,0.63,0.00}{##1}}}
\expandafter\def\csname PY@tok@gr\endcsname{\def\PY@tc##1{\textcolor[rgb]{1.00,0.00,0.00}{##1}}}
\expandafter\def\csname PY@tok@ge\endcsname{\let\PY@it=\textit}
\expandafter\def\csname PY@tok@gs\endcsname{\let\PY@bf=\textbf}
\expandafter\def\csname PY@tok@gp\endcsname{\let\PY@bf=\textbf\def\PY@tc##1{\textcolor[rgb]{0.00,0.00,0.50}{##1}}}
\expandafter\def\csname PY@tok@go\endcsname{\def\PY@tc##1{\textcolor[rgb]{0.53,0.53,0.53}{##1}}}
\expandafter\def\csname PY@tok@gt\endcsname{\def\PY@tc##1{\textcolor[rgb]{0.00,0.27,0.87}{##1}}}
\expandafter\def\csname PY@tok@err\endcsname{\def\PY@bc##1{\setlength{\fboxsep}{0pt}\fcolorbox[rgb]{1.00,0.00,0.00}{1,1,1}{\strut ##1}}}
\expandafter\def\csname PY@tok@kc\endcsname{\let\PY@bf=\textbf\def\PY@tc##1{\textcolor[rgb]{0.00,0.50,0.00}{##1}}}
\expandafter\def\csname PY@tok@kd\endcsname{\let\PY@bf=\textbf\def\PY@tc##1{\textcolor[rgb]{0.00,0.50,0.00}{##1}}}
\expandafter\def\csname PY@tok@kn\endcsname{\let\PY@bf=\textbf\def\PY@tc##1{\textcolor[rgb]{0.00,0.50,0.00}{##1}}}
\expandafter\def\csname PY@tok@kr\endcsname{\let\PY@bf=\textbf\def\PY@tc##1{\textcolor[rgb]{0.00,0.50,0.00}{##1}}}
\expandafter\def\csname PY@tok@bp\endcsname{\def\PY@tc##1{\textcolor[rgb]{0.00,0.50,0.00}{##1}}}
\expandafter\def\csname PY@tok@fm\endcsname{\def\PY@tc##1{\textcolor[rgb]{0.00,0.00,1.00}{##1}}}
\expandafter\def\csname PY@tok@vc\endcsname{\def\PY@tc##1{\textcolor[rgb]{0.10,0.09,0.49}{##1}}}
\expandafter\def\csname PY@tok@vg\endcsname{\def\PY@tc##1{\textcolor[rgb]{0.10,0.09,0.49}{##1}}}
\expandafter\def\csname PY@tok@vi\endcsname{\def\PY@tc##1{\textcolor[rgb]{0.10,0.09,0.49}{##1}}}
\expandafter\def\csname PY@tok@vm\endcsname{\def\PY@tc##1{\textcolor[rgb]{0.10,0.09,0.49}{##1}}}
\expandafter\def\csname PY@tok@sa\endcsname{\def\PY@tc##1{\textcolor[rgb]{0.73,0.13,0.13}{##1}}}
\expandafter\def\csname PY@tok@sb\endcsname{\def\PY@tc##1{\textcolor[rgb]{0.73,0.13,0.13}{##1}}}
\expandafter\def\csname PY@tok@sc\endcsname{\def\PY@tc##1{\textcolor[rgb]{0.73,0.13,0.13}{##1}}}
\expandafter\def\csname PY@tok@dl\endcsname{\def\PY@tc##1{\textcolor[rgb]{0.73,0.13,0.13}{##1}}}
\expandafter\def\csname PY@tok@s2\endcsname{\def\PY@tc##1{\textcolor[rgb]{0.73,0.13,0.13}{##1}}}
\expandafter\def\csname PY@tok@sh\endcsname{\def\PY@tc##1{\textcolor[rgb]{0.73,0.13,0.13}{##1}}}
\expandafter\def\csname PY@tok@s1\endcsname{\def\PY@tc##1{\textcolor[rgb]{0.73,0.13,0.13}{##1}}}
\expandafter\def\csname PY@tok@mb\endcsname{\def\PY@tc##1{\textcolor[rgb]{0.40,0.40,0.40}{##1}}}
\expandafter\def\csname PY@tok@mf\endcsname{\def\PY@tc##1{\textcolor[rgb]{0.40,0.40,0.40}{##1}}}
\expandafter\def\csname PY@tok@mh\endcsname{\def\PY@tc##1{\textcolor[rgb]{0.40,0.40,0.40}{##1}}}
\expandafter\def\csname PY@tok@mi\endcsname{\def\PY@tc##1{\textcolor[rgb]{0.40,0.40,0.40}{##1}}}
\expandafter\def\csname PY@tok@il\endcsname{\def\PY@tc##1{\textcolor[rgb]{0.40,0.40,0.40}{##1}}}
\expandafter\def\csname PY@tok@mo\endcsname{\def\PY@tc##1{\textcolor[rgb]{0.40,0.40,0.40}{##1}}}
\expandafter\def\csname PY@tok@ch\endcsname{\let\PY@it=\textit\def\PY@tc##1{\textcolor[rgb]{0.25,0.50,0.50}{##1}}}
\expandafter\def\csname PY@tok@cm\endcsname{\let\PY@it=\textit\def\PY@tc##1{\textcolor[rgb]{0.25,0.50,0.50}{##1}}}
\expandafter\def\csname PY@tok@cpf\endcsname{\let\PY@it=\textit\def\PY@tc##1{\textcolor[rgb]{0.25,0.50,0.50}{##1}}}
\expandafter\def\csname PY@tok@c1\endcsname{\let\PY@it=\textit\def\PY@tc##1{\textcolor[rgb]{0.25,0.50,0.50}{##1}}}
\expandafter\def\csname PY@tok@cs\endcsname{\let\PY@it=\textit\def\PY@tc##1{\textcolor[rgb]{0.25,0.50,0.50}{##1}}}

\def\PYZbs{\char`\\}
\def\PYZus{\char`\_}
\def\PYZob{\char`\{}
\def\PYZcb{\char`\}}
\def\PYZca{\char`\^}
\def\PYZam{\char`\&}
\def\PYZlt{\char`\<}
\def\PYZgt{\char`\>}
\def\PYZsh{\char`\#}
\def\PYZpc{\char`\%}
\def\PYZdl{\char`\$}
\def\PYZhy{\char`\-}
\def\PYZsq{\char`\'}
\def\PYZdq{\char`\"}
\def\PYZti{\char`\~}
% for compatibility with earlier versions
\def\PYZat{@}
\def\PYZlb{[}
\def\PYZrb{]}
\makeatother


    % Exact colors from NB
    \definecolor{incolor}{rgb}{0.0, 0.0, 0.5}
    \definecolor{outcolor}{rgb}{0.545, 0.0, 0.0}



    
    % Prevent overflowing lines due to hard-to-break entities
    \sloppy 
    % Setup hyperref package
    \hypersetup{
      breaklinks=true,  % so long urls are correctly broken across lines
      colorlinks=true,
      urlcolor=urlcolor,
      linkcolor=linkcolor,
      citecolor=citecolor,
      }
    % Slightly bigger margins than the latex defaults
    
    \geometry{verbose,tmargin=1in,bmargin=1in,lmargin=1in,rmargin=1in}
    
    

    \begin{document}
    
    
    \maketitle
    
    

    
    \section{Assignment 0}\label{assignment-0}

\begin{itemize}
\tightlist
\item
  Aman Joshi (2018201097)
\end{itemize}

    \begin{Verbatim}[commandchars=\\\{\}]
{\color{incolor}In [{\color{incolor}246}]:} \PY{k+kn}{import} \PY{n+nn}{numpy} \PY{k}{as} \PY{n+nn}{np}
          \PY{k+kn}{import} \PY{n+nn}{cv2} \PY{k}{as} \PY{n+nn}{cv}
          \PY{k+kn}{import} \PY{n+nn}{os}
\end{Verbatim}


    \subsection{Question 1 (video↔ images)}\label{question-1-video-images}

\subsubsection{Convert video to images}\label{convert-video-to-images}

I've used cv2 for converting video to images. The function is taking
video path and desetination folder as arguments. * Read the video frame
by frame using cv.VideoCapture() * Then for each succesfull read of the
frame write it to the destination folder using cv.imwrite.

    \begin{Verbatim}[commandchars=\\\{\}]
{\color{incolor}In [{\color{incolor}247}]:} \PY{k}{def} \PY{n+nf}{convert\PYZus{}video\PYZus{}to\PYZus{}image}\PY{p}{(}\PY{n}{video\PYZus{}name}\PY{p}{,} \PY{n}{folder\PYZus{}path}\PY{p}{)}\PY{p}{:}
              \PY{n}{cap} \PY{o}{=} \PY{n}{cv}\PY{o}{.}\PY{n}{VideoCapture}\PY{p}{(}\PY{n}{video\PYZus{}name}\PY{p}{)}
              \PY{n}{cnt} \PY{o}{=} \PY{l+m+mi}{1}
              \PY{k}{try}\PY{p}{:}
                  \PY{k}{while}\PY{p}{(}\PY{n}{cap}\PY{o}{.}\PY{n}{isOpened}\PY{p}{(}\PY{p}{)}\PY{p}{)}\PY{p}{:}
                      \PY{n}{ret}\PY{p}{,} \PY{n}{frame} \PY{o}{=} \PY{n}{cap}\PY{o}{.}\PY{n}{read}\PY{p}{(}\PY{p}{)}
                      \PY{n}{cv}\PY{o}{.}\PY{n}{imshow}\PY{p}{(}\PY{l+s+s1}{\PYZsq{}}\PY{l+s+s1}{frame}\PY{l+s+s1}{\PYZsq{}}\PY{p}{,} \PY{n}{frame}\PY{p}{)}
                      \PY{n}{key} \PY{o}{=} \PY{n}{cv}\PY{o}{.}\PY{n}{waitKey}\PY{p}{(}\PY{l+m+mi}{1}\PY{p}{)}
                      \PY{k}{if} \PY{p}{(}\PY{n}{key} \PY{o}{==} \PY{n+nb}{ord}\PY{p}{(}\PY{l+s+s1}{\PYZsq{}}\PY{l+s+s1}{q}\PY{l+s+s1}{\PYZsq{}}\PY{p}{)}\PY{p}{)}\PY{p}{:}
                          \PY{k}{break}
                      \PY{n}{cv}\PY{o}{.}\PY{n}{imwrite}\PY{p}{(}\PY{n}{os}\PY{o}{.}\PY{n}{path}\PY{o}{.}\PY{n}{join}\PY{p}{(}\PY{n}{folder\PYZus{}path}\PY{p}{,} \PY{l+s+s1}{\PYZsq{}}\PY{l+s+s1}{frame}\PY{l+s+s1}{\PYZsq{}}\PY{o}{+}\PY{n+nb}{str}\PY{p}{(}\PY{n}{cnt}\PY{p}{)}\PY{o}{+}\PY{l+s+s1}{\PYZsq{}}\PY{l+s+s1}{.jpg}\PY{l+s+s1}{\PYZsq{}}\PY{p}{)}\PY{p}{,} \PY{n}{frame}\PY{p}{)}
                      \PY{n}{cnt} \PY{o}{+}\PY{o}{=} \PY{l+m+mi}{1}
                  \PY{n}{cap}\PY{o}{.}\PY{n}{release}\PY{p}{(}\PY{p}{)}
                  \PY{n}{cv}\PY{o}{.}\PY{n}{destroyAllWindows}\PY{p}{(}\PY{p}{)}
              \PY{k}{except}\PY{p}{:}
                  \PY{n}{cap}\PY{o}{.}\PY{n}{release}\PY{p}{(}\PY{p}{)}
                  \PY{n}{cv}\PY{o}{.}\PY{n}{destroyAllWindows}\PY{p}{(}\PY{p}{)}
\end{Verbatim}


    \begin{Verbatim}[commandchars=\\\{\}]
{\color{incolor}In [{\color{incolor}248}]:} \PY{n}{convert\PYZus{}video\PYZus{}to\PYZus{}image}\PY{p}{(}\PY{l+s+s1}{\PYZsq{}}\PY{l+s+s1}{cute.mp4}\PY{l+s+s1}{\PYZsq{}}\PY{p}{,} \PY{l+s+s1}{\PYZsq{}}\PY{l+s+s1}{result}\PY{l+s+s1}{\PYZsq{}}\PY{p}{)}
\end{Verbatim}


    \subsubsection{Convert video to images}\label{convert-video-to-images}

I've used cv2 for converting iamges to video. The function is taking
image folder and desetination video as arguments. * Read the all file
name using os.listdir also sort them (if theya re part of some video and
been stored in some ordered fashioned). * Create a Video Writer object
for writing video. * Pass f.p.s. and size of frame to it. * Read each
image using cv.imread() and write it to video writer object. * Release
video object.

    \begin{Verbatim}[commandchars=\\\{\}]
{\color{incolor}In [{\color{incolor}249}]:} \PY{k}{def} \PY{n+nf}{convert\PYZus{}images\PYZus{}to\PYZus{}video}\PY{p}{(}\PY{n}{image\PYZus{}folder}\PY{p}{,} \PY{n}{video\PYZus{}path}\PY{p}{)}\PY{p}{:}
              \PY{n}{only\PYZus{}files} \PY{o}{=} \PY{p}{[}\PY{n}{f} \PY{k}{for} \PY{n}{f} \PY{o+ow}{in} \PY{n}{os}\PY{o}{.}\PY{n}{listdir}\PY{p}{(}\PY{n}{image\PYZus{}folder}\PY{p}{)} \PY{k}{if} \PY{n}{os}\PY{o}{.}\PY{n}{path}\PY{o}{.}\PY{n}{isfile}\PY{p}{(}\PY{n}{os}\PY{o}{.}\PY{n}{path}\PY{o}{.}\PY{n}{join}\PY{p}{(}\PY{n}{image\PYZus{}folder}\PY{p}{,} \PY{n}{f}\PY{p}{)}\PY{p}{)}\PY{p}{]}
              \PY{n}{fourcc} \PY{o}{=} \PY{n}{cv}\PY{o}{.}\PY{n}{VideoWriter\PYZus{}fourcc}\PY{p}{(}\PY{o}{*}\PY{l+s+s1}{\PYZsq{}}\PY{l+s+s1}{XVID}\PY{l+s+s1}{\PYZsq{}}\PY{p}{)}
              \PY{n}{out} \PY{o}{=} \PY{n}{cv}\PY{o}{.}\PY{n}{VideoWriter}\PY{p}{(}\PY{n}{video\PYZus{}path}\PY{p}{,}\PY{n}{fourcc}\PY{p}{,} \PY{l+m+mi}{20}\PY{p}{,} \PY{p}{(}\PY{l+m+mi}{1280}\PY{p}{,}\PY{l+m+mi}{720}\PY{p}{)}\PY{p}{)}
              \PY{k}{try}\PY{p}{:}
                  \PY{k}{for} \PY{n}{f} \PY{o+ow}{in} \PY{n}{only\PYZus{}files}\PY{p}{:}
                      \PY{n}{frame} \PY{o}{=} \PY{n}{cv}\PY{o}{.}\PY{n}{imread}\PY{p}{(}\PY{n}{os}\PY{o}{.}\PY{n}{path}\PY{o}{.}\PY{n}{join}\PY{p}{(}\PY{n}{image\PYZus{}folder}\PY{p}{,}\PY{n}{f}\PY{p}{)}\PY{p}{)}
                      \PY{n}{out}\PY{o}{.}\PY{n}{write}\PY{p}{(}\PY{n}{frame}\PY{p}{)}
                  \PY{n}{out}\PY{o}{.}\PY{n}{release}\PY{p}{(}\PY{p}{)}
                  \PY{n}{cv}\PY{o}{.}\PY{n}{destroyAllWindows}\PY{p}{(}\PY{p}{)}
              \PY{k}{except}\PY{p}{:}
                  \PY{n}{out}\PY{o}{.}\PY{n}{release}\PY{p}{(}\PY{p}{)}
                  \PY{n}{cv}\PY{o}{.}\PY{n}{destroyAllWindows}\PY{p}{(}\PY{p}{)}
\end{Verbatim}


    \begin{Verbatim}[commandchars=\\\{\}]
{\color{incolor}In [{\color{incolor}250}]:} \PY{n}{convert\PYZus{}images\PYZus{}to\PYZus{}video}\PY{p}{(}\PY{l+s+s1}{\PYZsq{}}\PY{l+s+s1}{result}\PY{l+s+s1}{\PYZsq{}}\PY{p}{,} \PY{l+s+s1}{\PYZsq{}}\PY{l+s+s1}{result.mp4}\PY{l+s+s1}{\PYZsq{}}\PY{p}{)}
\end{Verbatim}


    \subsection{Question 2 (Capturing
Images)}\label{question-2-capturing-images}

I've used cv2 for capturing stream from webcam of laptop. The function
is taking destination folder where screen shots will get saved (By
default screen). * Read the stream frame by frame. * Show image for 1ms.
* Read the key pressed during the display time i.e. 1ms * Whenever 'c'
is pressed it will save the current frame at the specified destination
using cv.imwrite()

    \begin{Verbatim}[commandchars=\\\{\}]
{\color{incolor}In [{\color{incolor}254}]:} \PY{k}{def} \PY{n+nf}{camera}\PY{p}{(}\PY{n}{screen\PYZus{}folder} \PY{o}{=} \PY{l+s+s2}{\PYZdq{}}\PY{l+s+s2}{screen}\PY{l+s+s2}{\PYZdq{}}\PY{p}{)}\PY{p}{:}
              \PY{n}{cap} \PY{o}{=} \PY{n}{cv}\PY{o}{.}\PY{n}{VideoCapture}\PY{p}{(}\PY{l+m+mi}{0}\PY{p}{)}
              \PY{n}{cnt} \PY{o}{=} \PY{l+m+mi}{0}
              \PY{k}{try}\PY{p}{:}
                  \PY{k}{while} \PY{k+kc}{True}\PY{p}{:}
                      \PY{n}{ret}\PY{p}{,} \PY{n}{frame} \PY{o}{=} \PY{n}{cap}\PY{o}{.}\PY{n}{read}\PY{p}{(}\PY{p}{)}
          \PY{c+c1}{\PYZsh{}             frame = cv.flip(frame, 1)}
                      \PY{n}{cv}\PY{o}{.}\PY{n}{imshow}\PY{p}{(}\PY{l+s+s1}{\PYZsq{}}\PY{l+s+s1}{frame}\PY{l+s+s1}{\PYZsq{}}\PY{p}{,} \PY{n}{frame}\PY{p}{[}\PY{p}{:}\PY{p}{:}\PY{p}{]}\PY{p}{)}
                      \PY{n}{key} \PY{o}{=} \PY{n}{cv}\PY{o}{.}\PY{n}{waitKey}\PY{p}{(}\PY{l+m+mi}{1}\PY{p}{)}
                      \PY{k}{if} \PY{p}{(}\PY{n}{key} \PY{o}{==} \PY{n+nb}{ord}\PY{p}{(}\PY{l+s+s1}{\PYZsq{}}\PY{l+s+s1}{c}\PY{l+s+s1}{\PYZsq{}}\PY{p}{)}\PY{p}{)}\PY{p}{:}
                          \PY{n}{cnt} \PY{o}{=} \PY{n}{cnt}\PY{o}{+}\PY{l+m+mi}{1}
                          \PY{n}{cv}\PY{o}{.}\PY{n}{imwrite}\PY{p}{(}\PY{n}{os}\PY{o}{.}\PY{n}{path}\PY{o}{.}\PY{n}{join}\PY{p}{(}\PY{n}{screen\PYZus{}folder} \PY{p}{,} \PY{n+nb}{str}\PY{p}{(}\PY{n}{cnt}\PY{p}{)} \PY{o}{+} \PY{l+s+s2}{\PYZdq{}}\PY{l+s+s2}{.jpg}\PY{l+s+s2}{\PYZdq{}}\PY{p}{)}\PY{p}{,} \PY{n}{frame}\PY{p}{)}
                      \PY{k}{elif} \PY{p}{(}\PY{n}{key} \PY{o}{==} \PY{n+nb}{ord}\PY{p}{(}\PY{l+s+s1}{\PYZsq{}}\PY{l+s+s1}{q}\PY{l+s+s1}{\PYZsq{}}\PY{p}{)}\PY{p}{)}\PY{p}{:}
                          \PY{k}{break}
                  \PY{n}{cap}\PY{o}{.}\PY{n}{release}\PY{p}{(}\PY{p}{)}
                  \PY{n}{cv}\PY{o}{.}\PY{n}{destroyAllWindows}\PY{p}{(}\PY{p}{)}
              \PY{k}{except}\PY{p}{:}
                  \PY{n}{cap}\PY{o}{.}\PY{n}{release}\PY{p}{(}\PY{p}{)}
                  \PY{n}{cv}\PY{o}{.}\PY{n}{destroyAllWindows}\PY{p}{(}\PY{p}{)}
\end{Verbatim}


    \begin{Verbatim}[commandchars=\\\{\}]
{\color{incolor}In [{\color{incolor}255}]:} \PY{n+nb}{print}\PY{p}{(}\PY{l+s+s1}{\PYZsq{}}\PY{l+s+s1}{Press }\PY{l+s+s1}{\PYZdq{}}\PY{l+s+s1}{q}\PY{l+s+s1}{\PYZdq{}}\PY{l+s+s1}{ for exiting}\PY{l+s+s1}{\PYZsq{}}\PY{p}{)}
          \PY{n+nb}{print}\PY{p}{(}\PY{l+s+s1}{\PYZsq{}}\PY{l+s+s1}{Press }\PY{l+s+s1}{\PYZdq{}}\PY{l+s+s1}{c}\PY{l+s+s1}{\PYZdq{}}\PY{l+s+s1}{ for taking snapshot}\PY{l+s+s1}{\PYZsq{}}\PY{p}{)}
          \PY{n}{camera}\PY{p}{(}\PY{l+s+s2}{\PYZdq{}}\PY{l+s+s2}{screen}\PY{l+s+s2}{\PYZdq{}}\PY{p}{)}
\end{Verbatim}


    \begin{Verbatim}[commandchars=\\\{\}]
Press "q" for exiting
Press "c" for taking snapshot

    \end{Verbatim}

    \subsection{Question 3 (Chroma Keying)}\label{question-3-chroma-keying}

Merge Two Videos, One with a foreground object with green background,
While the other video is of a Background.

\begin{itemize}
\tightlist
\item
  Read both videos frame by frame
\item
  Mark the green dominant regions of the frame and store them in alpha.
\item
  Alpha = 0 means that the region was green dominant (to be removed)
  while alpha = 1 means the region is to be kept.
\item
  Replace all the non green dominant region from the background image
  with the pixel values from the foreground image.
\item
  Save the frames to video with the above mentioned process.
\end{itemize}

    \begin{Verbatim}[commandchars=\\\{\}]
{\color{incolor}In [{\color{incolor}256}]:} \PY{k}{def} \PY{n+nf}{remove\PYZus{}green}\PY{p}{(}\PY{n}{img}\PY{p}{)}\PY{p}{:}
              \PY{n}{r} \PY{o}{=} \PY{n}{img}\PY{p}{[}\PY{p}{:}\PY{p}{,} \PY{p}{:}\PY{p}{,} \PY{l+m+mi}{0}\PY{p}{]}\PY{o}{/}\PY{l+m+mi}{1}
              \PY{n}{g} \PY{o}{=} \PY{n}{img}\PY{p}{[}\PY{p}{:}\PY{p}{,} \PY{p}{:}\PY{p}{,} \PY{l+m+mi}{1}\PY{p}{]}\PY{o}{/}\PY{l+m+mi}{1} 
              \PY{n}{b} \PY{o}{=} \PY{n}{img}\PY{p}{[}\PY{p}{:}\PY{p}{,} \PY{p}{:}\PY{p}{,} \PY{l+m+mi}{2}\PY{p}{]}\PY{o}{/}\PY{l+m+mi}{1}
              \PY{n}{red\PYZus{}vs\PYZus{}green} \PY{o}{=} \PY{p}{(}\PY{n}{r} \PY{o}{\PYZhy{}} \PY{n}{g}\PY{p}{)} \PY{o}{+} \PY{l+m+mi}{20}
              \PY{n}{blue\PYZus{}vs\PYZus{}green} \PY{o}{=} \PY{p}{(}\PY{n}{b} \PY{o}{\PYZhy{}} \PY{n}{g}\PY{p}{)} \PY{o}{+} \PY{l+m+mi}{20}
              \PY{n}{red\PYZus{}vs\PYZus{}green}\PY{p}{[}\PY{n}{red\PYZus{}vs\PYZus{}green} \PY{o}{\PYZlt{}} \PY{l+m+mi}{0}\PY{p}{]} \PY{o}{=} \PY{l+m+mi}{0}
              \PY{n}{blue\PYZus{}vs\PYZus{}green}\PY{p}{[}\PY{n}{blue\PYZus{}vs\PYZus{}green} \PY{o}{\PYZlt{}} \PY{l+m+mi}{0}\PY{p}{]} \PY{o}{=} \PY{l+m+mi}{0}
              
              \PY{n}{alpha} \PY{o}{=} \PY{p}{(}\PY{n}{red\PYZus{}vs\PYZus{}green} \PY{o}{+} \PY{n}{blue\PYZus{}vs\PYZus{}green}\PY{p}{)} 
              \PY{n}{alpha}\PY{p}{[}\PY{n}{alpha} \PY{o}{\PYZgt{}} \PY{l+m+mi}{50}\PY{p}{]} \PY{o}{=} \PY{l+m+mi}{255}
              \PY{n}{alpha} \PY{o}{=} \PY{n}{alpha}\PY{o}{/}\PY{l+m+mi}{255}\PY{p}{;}
              
              \PY{k}{return} \PY{n}{img}\PY{p}{,} \PY{n}{alpha}
\end{Verbatim}


    \begin{Verbatim}[commandchars=\\\{\}]
{\color{incolor}In [{\color{incolor}257}]:} \PY{k}{def} \PY{n+nf}{blend}\PY{p}{(}\PY{n}{bg}\PY{p}{,} \PY{n}{img}\PY{p}{,} \PY{n}{alpha}\PY{p}{)}\PY{p}{:}
          \PY{c+c1}{\PYZsh{}     print(bg.shape, img.shape)}
              \PY{n}{pixel\PYZus{}preserve} \PY{o}{=} \PY{p}{(}\PY{n}{alpha} \PY{o}{\PYZgt{}} \PY{l+m+mi}{0}\PY{p}{)}
          \PY{c+c1}{\PYZsh{}     print(pixel\PYZus{}preserve)}
          \PY{c+c1}{\PYZsh{}     print(alpha.shape, bg.shape)}
          \PY{c+c1}{\PYZsh{}     alpha = alpha.T}
          \PY{c+c1}{\PYZsh{}     alpha = np.asarray([alpha]*bg.shape[\PYZhy{}1])}
          \PY{c+c1}{\PYZsh{}     alpha = alpha.T}
          \PY{c+c1}{\PYZsh{}     print(alpha.shape)}
          \PY{c+c1}{\PYZsh{}     unique, counts = np.unique(alpha, return\PYZus{}counts=True)}
          \PY{c+c1}{\PYZsh{}     print(unique, counts)}
          \PY{c+c1}{\PYZsh{}     bg = (bg*(alpha) + (1\PYZhy{}alpha)*img)}
              \PY{n}{bg}\PY{p}{[}\PY{n}{pixel\PYZus{}preserve}\PY{p}{]} \PY{o}{=} \PY{n}{img}\PY{p}{[}\PY{n}{pixel\PYZus{}preserve}\PY{p}{]}
              \PY{k}{return} \PY{n}{bg}
\end{Verbatim}


    \begin{Verbatim}[commandchars=\\\{\}]
{\color{incolor}In [{\color{incolor}258}]:} \PY{n}{cap1} \PY{o}{=} \PY{n}{cv}\PY{o}{.}\PY{n}{VideoCapture}\PY{p}{(}\PY{l+s+s1}{\PYZsq{}}\PY{l+s+s1}{foreground.mp4}\PY{l+s+s1}{\PYZsq{}}\PY{p}{)}
          \PY{n}{cap2} \PY{o}{=} \PY{n}{cv}\PY{o}{.}\PY{n}{VideoCapture}\PY{p}{(}\PY{l+s+s1}{\PYZsq{}}\PY{l+s+s1}{background.mp4}\PY{l+s+s1}{\PYZsq{}}\PY{p}{)}
          
          \PY{n}{out} \PY{o}{=} \PY{n}{cv}\PY{o}{.}\PY{n}{VideoWriter}\PY{p}{(}\PY{l+s+s1}{\PYZsq{}}\PY{l+s+s1}{result\PYZus{}q3.mp4}\PY{l+s+s1}{\PYZsq{}}\PY{p}{,}\PY{o}{\PYZhy{}}\PY{l+m+mi}{1}\PY{p}{,} \PY{l+m+mi}{20}\PY{p}{,} \PY{p}{(}\PY{l+m+mi}{1280}\PY{p}{,}\PY{l+m+mi}{720}\PY{p}{)}\PY{p}{)}
          
          \PY{k}{try}\PY{p}{:}
              \PY{k}{while}\PY{p}{(}\PY{n}{cap1}\PY{o}{.}\PY{n}{isOpened}\PY{p}{(}\PY{p}{)} \PY{o+ow}{and} \PY{n}{cap2}\PY{o}{.}\PY{n}{isOpened}\PY{p}{(}\PY{p}{)}\PY{p}{)}\PY{p}{:}
                  \PY{n}{ret}\PY{p}{,} \PY{n}{fg} \PY{o}{=} \PY{n}{cap1}\PY{o}{.}\PY{n}{read}\PY{p}{(}\PY{p}{)}
                  \PY{n}{ret}\PY{p}{,} \PY{n}{bg} \PY{o}{=} \PY{n}{cap2}\PY{o}{.}\PY{n}{read}\PY{p}{(}\PY{p}{)}   
                  
                  \PY{n}{h}\PY{p}{,}\PY{n}{w} \PY{o}{=} \PY{n}{fg}\PY{o}{.}\PY{n}{shape}\PY{p}{[}\PY{p}{:}\PY{l+m+mi}{2}\PY{p}{]}
                  \PY{n}{bg} \PY{o}{=} \PY{n}{cv}\PY{o}{.}\PY{n}{resize}\PY{p}{(}\PY{n}{bg}\PY{p}{,} \PY{p}{(}\PY{n}{w}\PY{p}{,}\PY{n}{h}\PY{p}{)}\PY{p}{)}
                  
                  \PY{n}{fg}\PY{p}{,} \PY{n}{alpha} \PY{o}{=} \PY{n}{remove\PYZus{}green}\PY{p}{(}\PY{n}{fg}\PY{p}{)}
                  \PY{n}{bg} \PY{o}{=} \PY{n}{blend}\PY{p}{(}\PY{n}{bg}\PY{p}{,} \PY{n}{fg}\PY{p}{,} \PY{n}{alpha}\PY{p}{)}
                  
                  \PY{n}{cv}\PY{o}{.}\PY{n}{imshow}\PY{p}{(}\PY{l+s+s1}{\PYZsq{}}\PY{l+s+s1}{merged}\PY{l+s+s1}{\PYZsq{}}\PY{p}{,} \PY{n}{bg}\PY{p}{)}
                  \PY{n}{key} \PY{o}{=} \PY{n}{cv}\PY{o}{.}\PY{n}{waitKey}\PY{p}{(}\PY{l+m+mi}{1}\PY{p}{)}
                  \PY{k}{if} \PY{p}{(}\PY{n}{key} \PY{o}{==} \PY{n+nb}{ord}\PY{p}{(}\PY{l+s+s1}{\PYZsq{}}\PY{l+s+s1}{q}\PY{l+s+s1}{\PYZsq{}}\PY{p}{)}\PY{p}{)}\PY{p}{:}
                          \PY{k}{break}
                          
                  \PY{n}{out}\PY{o}{.}\PY{n}{write}\PY{p}{(}\PY{n}{bg}\PY{p}{)}
              
              \PY{n}{out}\PY{o}{.}\PY{n}{release}\PY{p}{(}\PY{p}{)}
              \PY{n}{cap1}\PY{o}{.}\PY{n}{release}\PY{p}{(}\PY{p}{)}
              \PY{n}{cap2}\PY{o}{.}\PY{n}{release}\PY{p}{(}\PY{p}{)}
              \PY{n}{cv}\PY{o}{.}\PY{n}{destroyAllWindows}\PY{p}{(}\PY{p}{)}
          \PY{k}{except}\PY{p}{:}
              \PY{n}{cap1}\PY{o}{.}\PY{n}{release}\PY{p}{(}\PY{p}{)}
              \PY{n}{cap2}\PY{o}{.}\PY{n}{release}\PY{p}{(}\PY{p}{)}
              \PY{n}{cv}\PY{o}{.}\PY{n}{destroyAllWindows}\PY{p}{(}\PY{p}{)}
\end{Verbatim}


    \subsubsection{Learnings}\label{learnings}

\begin{itemize}
\tightlist
\item
  Learn how to open, save, show images using opencv.
\item
  Process videos frame by frame and treat each frame as images.
\item
  Learn Chroma Keying
\end{itemize}

    \subsubsection{Challanges}\label{challanges}

\begin{itemize}
\tightlist
\item
  Combining images to make vide. Unordered reading using os.listdir().
\item
  By doing operations on the color matrix lead to overflow. Thus
  changing its data type by doing some operations, like dividing by 1.
\item
  Hard to figure out the threshold, allowing upto which limit pixels
  from the foreground should be permitted.
\item
  Installing opencv on python3
\end{itemize}


    % Add a bibliography block to the postdoc
    
    
    
    \end{document}
